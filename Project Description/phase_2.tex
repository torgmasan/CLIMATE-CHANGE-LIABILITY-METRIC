\documentclass[fontsize=11pt]{article}
\usepackage{amsmath}
\usepackage[utf8]{inputenc}
\usepackage[margin=0.75in]{geometry}

\title{CSC110 Project Proposal: Country-wise Financial Responsibility for Climate Change}
\author{Ansh Jain, Nikhil Sreekumar, Sidharth Sachdev}
\date{Monday, December 14, 2020}

\begin{document}
    \maketitle

    \section*{Problem Description and Research Question}

    Change has been exponentially quick in the last century, and it certainly has not been kind to the climate of our world. Instead of guiding us away from the brink of destruction, many of our leaders have steered us right to the tipping point. Our problem is that some very privileged people are allowed to make decisions that adversely affect the entire planet. We have decided to research how various factors contribute to the individual impact of a nation on the worsening climate of the earth. We want to quantitatively find out how much each country is responsible for climate change.\newline

    Hence our research question is,

    \textbf{What is each country’s responsibility, in monetary value, towards confronting the looming climate emergency based on certain factors, ranging from the country’s effect on the earth’s climate to its fiscal ability?}\newline

    The world runs on money, and any form of commitment on a global front must start off with a budget. This resulted in us choosing a monetary approach to answer the research question. However, the aim of the project is not realized through calculating this aforementioned budget, a role imposed upon the user, but is done so by producing a fair distribution of this budget amongst the nations of the world.\newline

    We intend to go about doing so by creating a desktop application that allows the user to gain this information in whichever way is more preferable: comparing nations in a choropleth map or analyzing each country separately. It is also imperative that the application has a clean and understandable User Interface so that the user is aware of the expected input as well as the features that come with it.\newline


    A major feature of the application is providing the user with the freedom to manipulate the datasets, weightage, and factors while using a set of standard underlying equations. There is a default directory with datasets that the users can use, but they are encouraged to delete or replace them with datasets of their own before running the program. The weight of each factor, the budget, and the year of analysis can be directly altered while running the application. Together, this customizability allows for an interactive application that can produce several hypothetical budget distributions in a systematic, unbiased manner


    \section*{Dataset Description}

    There are two directories in which datasets are stored. 'Responsibility Datasets' is where the user enters datasets as per what best fits their idea of assigning responsibility to each nation state. 'Constant Datasets' cannot be altered by the user since they are standard values in the computational formula.


    \begin{center}
        CONSTANT DATASETS
    \end{center}

    \begin{enumerate}


        \item [1.] GDP is a dataset of the CSV format which provides basic information about each country such as GDP in from 1960 to 2017. The data has been compiled by the US government.

        \item [2.]  countries\_codes\_and\_coordinates contain information regarding location and ISO 2 and 3 codes. The country name and ISO code are used. Dataset obtained from a Github repository under username Tadast.

    \end{enumerate} \newpage

    \begin{center}
        RESPONSIBILITY DATASETS (DEFAULT)
    \end{center}

    \begin{enumerate}
        \item [1.] CO2 emission is a dataset of the CSV format which provides information on the emissions of Carbon Dioxide in metric tons based on each country from 1990 to 2016. This dataset has been sourced from data.world and was created by Oak Ridge National Laboratory.

        \item [2.] Percentage use of renewable energy is a dataset of the CSV format which provides information on the percentage of renewable energy used out of the total energy consumption per country from 1960 to 2015. The dataset has been sourced from the World Bank.

        \item [3.] CRI is a dataset of the CSV format which provides information on categories such as score, fatalities, losses in millions of dollars based on the effects of climate in each country in 2014. The source of the data set is GERMANWATCH.

        \item [4.] The same GDP dataset from Constant Datasets can also serve as a factor. Thus, it is also a part of the Responsibility Datasets

    \end{enumerate}


    Note: Each Dataset provided has data for a range of years. However, the data for CRI is from the year 2014 exclusively. Hence, the only common year available to the user for analysis is 2014. The user can change this by editing the directory containing datasets as required. The user’s preference and available data may differ and they have the freedom to input datasets in their best interest. This is elaborated further in the Computational Plan.


    \section*{Computational Plan}
    \textbf{NIKHIL edit this first paragraph(if required) followed by your on description of gathering data like dataset\_utilities etc.}
    \newline
    The application first takes in user input for a year to Analyze. The user is provided a drop-down of
    possible years to choose from(only 2014 until the CRI dataset is removed from the Responsibility Datasets directory).
    The computation to find the possible years requires checking if there
    is a column for that year in each input dataset. From this preliminary list of candidate years,
    those which contain no data for the valid countries in any input dataset are also of no use and
    are removed from the drop-down.\newline

    \begin{enumerate}
        \item [1.] Mention precondition of over 1 million
        \item [2.] all constants must be greater than 0 and equal to 100
    \end{enumerate}

    As the application is executed, the map\_win\_open method in Application.Functionalities.user\_input calls the run
    method in Map.py which is a public method that displays a choropleth map giving information about the
    budget, budget percentage, name of the country, its iso-code and each country’s data for every
    factor. This data can be viewed by hovering over the country on the map; this feature is executed
    using the package pandas’ DataFame method and plotly.express. The budget and budget percentage for each
    country are obtained by calling the private\_plot method that takes in the required information and
    calls the budget\_details method in computation.py to return a dictionary mapping each country code to a
    corresponding tuple of the budget and budget percentage.\newline

    budget\_details is a public method that calculates the appropriate amount in US dollars to be allocated to
    each country and returns a tuple containing both budget and budget percentage for that specific country
    based on the formula,
    \[
        Budget(A) = Responsibility(A) \cdot Total \_ Budget
        GDP \% = \frac{Budget(A)}{Total \, Budget} \cdot 100
    \]
    It does this by calling the private method \_responsibility. It carries out
    the major chunk of the calculation by first checking whether there is any unavailable data for a country in a
    specific factor (this is only if the country has data for at least one of the factors). If this is the
    case, then the country is removed from the dict and not displayed on the map. \newline

    \_responsibility also calls private methods \_negative\_calculation and \_positive\_calculation for each factor based on
    their relation of ‘direct’ or ‘inverse’ as input by the user.\newline
    \begin{enumerate}
        \item [-] if the word 'direct' is used, it means that the factor is positively correlated to climate change like
        GDP, Carbon Dioxide Emissions, and Climate Risk Index. It uses the formula,
        \[
            \frac{Country's data}{Total Data}
        \]

        \item [-] if the word 'inverse' is used, it means that the factor is negatively correlated to climate change like
        Renewable Energy.
        \[
            \frac{Total \, data - Country's \, data}{\displaystyle\sum_{i \in Country \, data} Total \, data - i }
        \]
    \end{enumerate}
    The result from these calculations(point out of 1) are then multiplied with the corresponding weight of the factor.
    This is done for each factor and their sum(out of 100) gives the responsibility of that country.
    \[
        Responsibility = \alpha F1 + \beta F2 + \gamma F3 + \delta F4
    \]
    A way to crosscheck these values
    is if the responsibilities of all countries are added, we get 100\%. Hence we know that the computation is right.
    \newline
    Using these methods, we were able to come up with an implementation for our plan and display the
    choropleth map along with the information of each country when the user hovers over that region.\newline



    \section*{References}

    TODO

\end{document}
