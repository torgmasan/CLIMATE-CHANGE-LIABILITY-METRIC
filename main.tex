\documentclass[fontsize=11pt]{article}
\usepackage{amsmath}
\usepackage[utf8]{inputenc}
\usepackage[margin=0.75in]{geometry}

\title{CSC110 Project Proposal: IT'S TIME TO KNOW}
\author{Sidharth Sachdev, Ansh Jain, Nikhil Sreekumar}
\date{Friday, November 6, 2020}

\begin{document}
\maketitle

\section*{Problem Description and Research Question}

Change has been exponentially quick in the last century, and it certainly has not been kind to the climate of our world. Instead of guiding us away from the brink of destruction, many of our leaders have steered us right to the tipping point; it is very difficult for us now to return to the wildlife-rich and climate-healthy world that we had before. Our problem is that some very privileged people are allowed to take decisions that adversely affect the entire planet.
\\ We have decided to research how various factors contribute to the individual impact of a nation on the worsening climate of the earth. We want to quantitatively find out how a country is responsible for climate change and discuss the need for immediate action by countries (especially those which are influencing climate change in a significantly larger proportion).

\section*{Dataset Description}

1. CO2 emissions is a dataset of the csv format which provides information on the emissions of Carbon Dioxide per capita in metric tons based on each country from 1960 to 2014. This dataset has been sourced from data.world and was created by Eva Murray in 2019. For the purpose of the project, only data from 2014 has been used. \\
2. Percentage of use of renewable energy is a dataset of the csv format which provides information on the percentage of  renewable energy used out of the total energy consumption per country from 1990 to 2015. The dataset has been sourced from the World Bank. For the purpose of this project, only data from 2015 has been used. \\
3. Counties of the world/GDP per capita is a dataset of the csv format which provides basic information about each country like its GDP per capita in dollars, population, area and more from 1970 to 2017. The data has been compiled by the US government. For the purpose of this project, only data for GDP per capita in dollars has been used. \\
4. CRI is a dataset of the csv format which provides information on categories like score, fatalities, losses in millions of dollars based on the effects of climate in each country in 2018. The source of the data set is GERMANWATCH. For the purpose of this assignment we will be only using the Climate Risk Index score. 

\section*{Computational Plan}

TODO

\section*{References}

TODO

% NOTE: LaTeX does have a built-in way of generating references automatically,
% but it's a bit tricky to use so we STRONGLY recommend writing your references
% manually, using a standard academic format like APA or MLA.
% (E.g., https://owl.purdue.edu/owl/research_and_citation/apa_style/apa_formatting_and_style_guide/general_format.html)

\end{document}
